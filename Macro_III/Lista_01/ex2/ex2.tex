\documentclass{article}

\usepackage[brazil]{babel}
\usepackage[utf8]{inputenc}
\usepackage{amsmath}

\usepackage{listings}
\usepackage{mcode}
\usepackage[section]{placeins}

\usepackage{graphicx}

\usepackage{color} %red, green, blue, yellow, cyan, magenta, black, white
\definecolor{mygreen}{RGB}{28,172,0} % color values Red, Green, Blue
\definecolor{mylilas}{RGB}{170,55,241}

\begin{document}

\begin{flushleft}
FUNDAÇÃO GETÚLIO VARGAS \\

Escola de Pós-Graduação em Economia

Teoria Macroeconômica III

Professor: Ricardo de Oliveira Cavalcanti

Monitora: Kátia Aiko Nishiyama Alves

Alunos: Gustavo Bulhões e Samuel Barbosa
\end{flushleft}

\section*{Exercício 02}

Neste exercício temos o modelo clássico de crescimento econômico, cujo problema do planejador
é escolher sequências de consumo $\{c_t\}_{t=0}^{\infty}$ e de capital $\{k_t\}_{t=0}^{\infty}$ que resolvem

\begin{equation}
\begin{aligned}
& \max & & \sum_{t=0}^{\infty} \beta^t u(c_t) \\
& \text{s.a.} & &  c_t + k_{t+1} \leq f(k_t) + (1-\delta) k_t \\
& & &  k_{t+1} \geq 0, c_t \geq 0 \,\, \forall t \geq 0  \\
& & &  k_0 \text{ dado} \\
\end{aligned}
\end{equation}

com $f(k) = k^\alpha$ e $u(c) = \frac{c^{1-\gamma}}{1-\gamma}$.


\subsection*{Item (i)}

Observe que $u(c)$ é monótona crescente em $c$ e, portanto satisfaz a propriedade de não saciedade local.
Logo vale a Lei de Walras, e podemos reescrever a primeira restrição com igualdade,
resolver para $c_t$ e substituir na função objetivo. Desta forma o problema se torna 

\begin{equation}
\begin{aligned}
& \max & & \sum_{t=0}^{\infty} \beta^t u(f(k_t) + (1-\delta) k_t - k_{t+1}) \\
& \text{s.a.} & &  k_{t+1} \geq 0, c_t \geq 0 \,\, \forall t \geq 0  \\
& & &  k_0 \text{ dado} \\
\end{aligned}
\end{equation}

\subsection*{Item (ii)}

Reescrevemos o problema sequencial na forma recursiva, transformando-o na equação funcional

\begin{equation}
\begin{aligned}
V(k) = & \max_{k'} & & u(c) + \beta V(k') \\
& \text{s.a.} & &  c + k' = f(k) + (1-\delta) k \\
& & &  k' \geq 0, c \geq 0 \,\, \forall t \geq 0  \\
\end{aligned}
\end{equation}

\subsection*{Item (iii)}

O operador de Bellman associado à equação funcional obtida no item anterior é justamente

\begin{equation}
\begin{aligned}
T(V)(k) = & \max_{k'} & & u(c) + \beta V(k') \\
& \text{s.a.} & &  c + k' = f(k) + (1-\delta) k \\
& & &  k' \geq 0, c \geq 0 \,\, \forall t \geq 0  \\
\end{aligned}
\end{equation}

\subsection*{Item (iv)}

Vamos criar um grid para a variável de estado $k$ no intervalo $[0, 1.25 k_{ss}]$, em que $k_{ss}$ é o nível
de capital de estado estacionário. 

Resolvendo o lado direito da equação funcional $(3)$, já substituindo as funções dadas $u()$ e $f()$, 
obtemos a equação de Euler

$$ c^{-\gamma} = \beta c'^{-\gamma} [\alpha k'^{\alpha-1} + 1 - \delta].$$

No estado estacionário temos que $c' = c$ e $k' = k$. Substituindo na equação anterior obtemos

$$k_{ss} = \left( \frac{1 + \beta (\delta - 1)}{\alpha \beta} \right)^{\frac{1}{\alpha - 1}}.$$

\begin{lstlisting}
\end{lstlisting}

\end{document}